\documentclass[a4paper]{article}

\usepackage[utf8]{inputenc}
\usepackage[T1]{fontenc}
\usepackage{amsmath, amssymb, amsthm}
\usepackage{xcolor, makeidx}
\usepackage[breaklinks,hidelinks]{hyperref}
\usepackage[xcolor, makeidx, hyperref, notion]{knowledge}

\newtheorem{theorem}{Theorem}[section]
\newtheorem{notation}{Notation}[section]
\newtheorem{corollary}{Corollary}[theorem]
\newtheorem{lemma}[theorem]{Lemma}
\newtheorem{definition}{Definition}[section]
\newtheorem{example}{Example}[definition]
\newtheorem*{remark}{Remark}

\title{Knowledge package example}
\author{Jose Abel Castellanos Joo}
\date{\today}

\knowledge{notion, index=semigroup}
| semigroup
| semigroups
| Semigroup
| Semigroups
\knowledge{notion, index=monoid}
| monoid
| monoids
| Monoid
| Monoids
\knowledge{notion, index=group}
| group
| groups
| Group
| Groups
\knowledge{notion, index=abelian}
| abelian
| Abelian
| commutative
| Commutative
\knowledge{notion, index=subgroup}
| subgroup
| Subgroup
| subgroups
| Subgroups
\knowledge{notion, index=ring}
| ring
| rings
| Ring
| Rings
\knowledge{notion, index=commutative ring}
| commutative ring
\knowledge{notion, index=ring with identity}
| ring with identity

\makeindex

\begin{document}

\maketitle

\begin{abstract}
  Quick document showing some features of the knowledge package.
\end{abstract}

\section{Semigroups, Monoids and Groups} \cite{Hungerford1980}

\begin{definition}
  A \intro{semigroup} is a nonempty set $G$ together
  with a binary operation on $G$ which is
  \begin{itemize}
    \item associative: $a(bc) = (ab)c$ for all $a, b, c \in G$
  \end{itemize}

  a \intro{monoid} is a \kl{semigroup} $G$ which contains a 
  \begin{itemize}
    \item (two-sided) identity element $e \in G$ such that $ae = ea = a$ for all $a \in G$.
  \end{itemize}

  A \intro{group} is a \kl{monoid} $G$ such that

  \begin{itemize}
    \item for every $a \in G$ there exists a (two-sided) inverse element $a^{-1} \in G$
      such that $a^{-1}a = aa^{-1} = e$.
  \end{itemize}

  A \kl{semigroup} $G$ is said to be \intro{abelian} if its binary operation is
  \begin{itemize}
    \item commutative: $ab = ba$ for all $a, b \in G$.
  \end{itemize}
\end{definition}

\section{Homomorphisms and Subgroups}

\begin{definition}
  Let $G$ be a group and $H$ a nonempty
  subset that is closed under the product in $G$.
  If $H$ is itself a \kl{group} under the product
  of $G$, then $H$ is said to be a \intro{subgroup} 
  of $G$. This is denoted by $H < G$.
\end{definition}

\input{lorem}

\section{Rings and Homomorphism}

\begin{definition}
  A \intro{ring} is a nonempty set $R$
  together with two binary operations (usually denoted as addition (+) and multiplication) such that:
  \begin{itemize}
    \item $(R, +)$ is an \kl{commutative} \kl{group};
    \item $(ab)c = a(bc)$ for all $a, b, c \in R$ (associate multiplication)
    \item $a(b + c) = ab + ac$ and $(a+b)c = ac + bc$
      (left and right distributive laws).
  \end{itemize}

  If in addition:
  \begin{itemize}
    \item $ab = ba$ for all $a, b \in R$,
  \end{itemize}

  then $R$ is said to be a \intro{commutative ring}.
  If $R$ contains an elements $1_R$ suc that
  \begin{itemize}
    \item $1_R a = a 1_R = a$ for all $a \in R$,
  \end{itemize}

  then $R$ is said to be a \intro{ring with identity}.
\end{definition}

\bibliographystyle{plain}
\bibliography{references}

\end{document}
